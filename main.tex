\documentclass{book} %book,article,report,letter
\usepackage{amsmath} %宏包
\usepackage{enumerate}
\usepackage{amssymb}
\usepackage{ctex}
\usepackage{graphicx}
\usepackage{titlesec}
\usepackage{amsthm}
\usepackage{indentfirst}
\usepackage{geometry}
\usepackage{color}
%\geometry{left=2.0cm,right=2.0cm,top=2.5cm,bottom=2.5cm}
\usepackage{fancyhdr}
\pagestyle{plain}
\newtheorem{dfn}{定义}[chapter]
\newtheorem{axi}[dfn]{公理}
\newtheorem{thm}[dfn]{定理}
\newtheorem{pro}[dfn]{命题}
\newtheorem{lem}[dfn]{引理}
\newtheorem{col}[dfn]{推论}


%导言区
\begin{document}
\pagenumbering{Roman}
\Large

\title{\textbf{云学社数学习题解答集\\ Cloud Society Solutions to Texts in Mathematics}}  %%书名
\author{Cloud Society} %%作者

%\date{} %%如果没有这句,会生成时间
\maketitle
\large

\renewcommand{\contentsname}{\centerline{目录}}
\tableofcontents  %%生成目录

\pagenumbering{arabic}
\setlength{\parindent}{0pt}

\mainmatter

\part{数学基础}

\part{代数}
\chapter{Axler: Linear Algebra Done Right}
\input{algebra/Axler_LADR}

\chapter{赵春来、徐明曜:抽象代数}
\section{群、环、体、域的基本概念}

\subsection{预备知识}
\par \textbf{1}. 设$X$和$Y$是两个集合, $f:X \to Y$和$g:Y \to X$ 是两个映射. 若$g \circ f=id_X$, 则称$g$为$f$的一个左逆; 若$f \circ g=id_Y$, 则称$g$为$f$的一个右逆. 证明:
(1) $f$有左逆当且仅当$f$是单射;
(2) $f$有右逆当且仅当$f$是满射;
(3) $f$有逆当且仅当$f$是双射;
(4) 若$f$有左逆$g$及右逆$h$, 则$g=h$;
(5) 若$f$有逆, 则$f$的逆唯一, 记作$f^{-1}$;
(6) 若$f$有逆, 则$(f^{-1})^{-1}=f$.

\par 证明: (1) 必要性:设$g\circ f=id_X$, 若$f(x_1)=f(x_2)$,  $x_1,x_2\in X$, 则$x_1=g\circ f(x_1)=g\circ f(x_2)=x_2$, 故$f$为单射; 充分性:设$f$为单射, 令$g:Y\to X$
\begin{equation*}
g(y)=
\begin{cases}
f^{-1}(y),& y\in f(X)\\
\text{任意}a\in X, & y\notin f(X)
\end{cases}
\end{equation*}
则$g\circ f=id_X$.
(2) 必要性:设$f\circ g=id_Y$, $\forall y\in Y$, $f(g(y))=y$,故$f$为满射; 充分性:设$f$为满射, 令$g:Y\to X$, 对$y\in Y$, 任取$a_y\in f^{-1}(y)$, 令$g(y)=a_y$, 则$f\circ g=id_Y$.
(3) 必要性:由(1)(2)之必要性成立; 充分性: $\forall y\in Y$,$f^{-1}(y)$($y$的原像)唯一. 令$g(y)=f^{-1}(y)$,则$g$是$f$的逆.
(4) $g=g\circ(f\circ h)=(g\circ f)\circ h=h$.
(5) 由(4)即得.
(6) 由$f\circ f^{-1}=id_Y, f^{-1}\circ f=id_X$, $f$是$f^{-1}$的逆.

\par \textbf{2}. 举例说明等价关系定义中的三个条件相互独立.
\par 解: 考虑$\mathbb{Z}$上的如下关系:
(1)小于等于关系(不满足对称性);
(2)差的绝对值小于等于2(不满足传递性);
(3)乘积是非零平方数(不满足自反性).

\subsection{群的基本概念}
\subsection{环的基本概念}
\subsection{体、域的基本概念}

\chapter{徐竞、徐明曜:近世代数初步}
\section{群、环、体、域}

\subsection{半群与群}
\par \textbf{10}. 设群$G$的每个元素$a$都满足$a^2=e$. 证明: $G$是交换群.

\begin{proof}
对任意$a,b\in G$, 有
\begin{equation*}
a^2b^2=e=(ab)^2.
\end{equation*}
上式左乘$a^{-1}$,右乘$b^{-1}$即得$ab=ba$.
\end{proof}

\par \textbf{14}.
证明: 偶数阶群中必有元素$a\neq e$, 满足$a^2=e$.

\begin{proof}
在有限群上定义二元关系: $a=b$或$a=b^{-1}$. 可以验证此为等价关系,且每个等价类至多含两个元素. 考虑偶数阶群的等价类划分,则知含一个元素的等价类有偶数个,即存在$e$以外的元素逆等于它本身。
\end{proof}

\subsection{环}
\par \textbf{11}. 设环$R$的非零元素$a,b$满足$aba=0$. 证明: $a$是左零因子或右零因子.

\begin{proof}
若$ab=0$, 则$a$是左零因子;否则,由$(ab)a=0$知$a$是右零因子.
\end{proof}

\subsection{体和域}
\par 略

\section{群}
\subsection{对称群}

\par \textbf{12}. 证明: 
\begin{displaymath}
|A_n|=\frac{|S_n|}{2}.
\end{displaymath}

\begin{proof}
考虑$S_n$中全体奇置换到全体偶置换的映射$L_{(12)}:\alpha\mapsto(12)\alpha$, 其良定义性是显然的. 又全体偶置换到全体奇置换的$L_{(12)}$为其逆映射,知其为双射. 从而奇置换与偶置换数目相等.
\end{proof}

\subsection{子群、生成子群}

\par \textbf{3}.
设$G$是群,$g\in G$. 令$C_G(g)=\{x\in G| xg=gx\}$,称为$g$在$G$中的中心化子. 证明:$C_G(g)\le G$,且$Z(G)=\bigcap_{g\in G}C_G(g)$.

\begin{proof}
对$a,b\in C_G(g)$,有$xab=axb=abx$,又由$xa=ax$知$a^{-1}x=xa^{-1}$,从而$ab,a^{-1}\in C_G(g)$, $C_G(g)\le G$. 
\par $Z(G)=\{x\in G| xg=gx, \forall g\in G\}=\bigcap_{g\in G} \{x\in G| xg=gx\}=\bigcap_{g\in G}C_G(g)$.
\end{proof}

\par \textbf{8}.
设$G$是群,$H\le G,K\le G$,证明:$H\cup K \le G\Leftrightarrow H\le K$或$K\le H$.

\begin{proof}
充分性显然,下证必要性. 用反证法,设$H\backslash K \neq \varnothing$且$K\backslash H \neq \varnothing$,取$a\in H\backslash K$,$b\in K\backslash H$. 由$H\cup K$是子群,$a,b\in H\cup K$, 知$ab \in H\cup K$. 若$ab \in H$,推出$b=a^{-1}ab\in H$,矛盾. 若$ab \in K$,同理得$a\in K$,矛盾. 故$H \subseteq K$或$K \subseteq H$,自然有$H\le K$或$K\le H$.
\end{proof}

\subsection{陪集、Lagrange定理}
\par \textbf{7}. 设群$G$非平凡群,且没有非平凡的真子群. 证明: $G$是素数阶循环群.

\begin{proof}
由$G$非平凡群,存在元素$a\neq e$. 考虑$a$的生成子群$\langle a\rangle$,其至少有$a,e$两个元素,由条件只能有$\langle a\rangle=G$,即$G$是循环群. 
\par 若$G$为无限群,则同构于整数加法群,其有非平凡真子群;当$n$是合数阶循环群时,存在阶为$n$素因子的元素,其生成子群为$G$的非平凡真子群. 故$G$只能是素数阶循环群.
\end{proof}

\subsection{正规子群与商群}

\par \textbf{4}.
设$H$是群$G$的子群,$a\in G$,证明:$aHa^{-1}$是$G$的子群且$aHa^{-1}\cong H$.

\begin{proof}
$e=aea^{-1}\in aHa^{-1}$,故$aHa^{-1}$非空;对$h_1,h_2\in H$,$ah_1a^{-1}ah_2a^{-1}=ah_1h_2a^{-1}\in H$, $(ah_1a^{-1})^{-1}=ah^{-1}_1a^{-1}\in H$,故$aHa^{-1}$是子群. 
\par 考虑映射:
\begin{align*}
    \phi:H\rightarrow& \ aHa^{-1}\\
    h\mapsto&  \ aha^{-1}
\end{align*}
则由$aHa^{-1}$的定义知$\phi$是满射;由$\phi(h_1h_2)=ah_1h_2a^{-1}=ah_1a^{-1}ah_2a^{-1}=\phi(h_1)\phi(h_2)$知$\phi$是群同态,又由$aha^{-1}=e \Rightarrow h=a^{-1}a=e$,$\phi$是单射,故$\phi$是群同构.
\end{proof}

\par \textbf{5}. 设$H$是群$G$的子群,证明:若$|H|=k$, 且$G$只有一个$k$阶子群,则$H \unlhd G$.

\begin{proof}
对任意$a\in G$,由2.4.4题知$aHa^{-1}$是$G$的子群且$|aHa^{-1}|=k$. 若$aHa^{-1}\neq H$,与$G$只有一个$k$阶子群矛盾. 故$aHa^{-1}=H$对任意$a\in G$成立,$H$是$G$的正规子群.
\end{proof}

\subsection{同态、同态基本定理}
\subsection{同构定理}
\subsection{群的直积}
\subsection{群在集合上的作用}
\subsection{Sylow定理}

\part{几何与拓扑}


\part{组合学}
\chapter{Brualdi: Introductory Combinatorics}
\input{combinatorics/Brualdi_IC}

\part{概率论}


\part{计算数学}

\chapter{Sipser: Introduction to Theory of Computation}
\section{绪论}
\section{正则语言}
\section{上下文无关文法}
\section{Church-Turing论题}

\section{可判定性}

\par \textbf{3}. 证明$ALL_{DFA}$可判定.
\par 证明: 类似于$E_{DFA}$的判别, 若所有起始状态出发的可达状态均为接受状态则接受, 否则拒绝.

\section{可归约性}

\par \textbf{1}. 证明$EQ_{CFG}$不可判定.
\par 证明: 利用计算历史归约可证明$ALL_{CFG}$不可判定. 设图灵机$R$判定$EQ_{CFG}$, 构造图灵机$S$判定$ALL_{CFG}$:
\par S=``对于输入$<G>$,$G$是CFG:
\par \quad 1.在输入$<G,G_1>$上运行$R$, $G_1$是生成$\Sigma^*$的CFG;
\par \quad 2.$R$接受,则接受; $R$拒绝, 则拒绝.''
\par 这与$ALL_{CFG}$不可判定矛盾.

\par \textbf{2}. 证明$EQ_{CFG}$补图灵可识别.
\par 证明: 构造图灵机$S$识别$\overline{EQ_{CFG}}$:
\par S=``对于输入$<G_1,G_2>$,$G_1,G_2$是CFG: 遍历$\Sigma^*$中的字符串, 对每个字符串s, 调用判定$A_{CFG}$的图灵机R, 若R对于$<G_1,s>,<G_2,s>$恰好接受一个, 则接受.''


\section{可计算性理论高级专题}

\end{document}

