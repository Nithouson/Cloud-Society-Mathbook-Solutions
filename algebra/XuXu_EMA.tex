\section{群、环、体、域}

\subsection{半群与群}
\par \textbf{10}. 设群$G$的每个元素$a$都满足$a^2=e$. 证明: $G$是交换群.

\begin{proof}
对任意$a,b\in G$, 有
\begin{equation*}
a^2b^2=e=(ab)^2.
\end{equation*}
上式左乘$a^{-1}$,右乘$b^{-1}$即得$ab=ba$.
\end{proof}

\par \textbf{14}.
证明: 偶数阶群中必有元素$a\neq e$, 满足$a^2=e$.

\begin{proof}
在有限群上定义二元关系: $a=b$或$a=b^{-1}$. 可以验证此为等价关系,且每个等价类至多含两个元素. 考虑偶数阶群的等价类划分,则知含一个元素的等价类有偶数个,即存在$e$以外的元素逆等于它本身。
\end{proof}

\subsection{环}
\par \textbf{11}. 设环$R$的非零元素$a,b$满足$aba=0$. 证明: $a$是左零因子或右零因子.

\begin{proof}
若$ab=0$, 则$a$是左零因子;否则,由$(ab)a=0$知$a$是右零因子.
\end{proof}

\subsection{体和域}
\par 略

\section{群}
\subsection{对称群}

\par \textbf{12}. 证明: 
\begin{displaymath}
|A_n|=\frac{|S_n|}{2}.
\end{displaymath}

\begin{proof}
考虑$S_n$中全体奇置换到全体偶置换的映射$L_{(12)}:\alpha\mapsto(12)\alpha$, 其良定义性是显然的. 又全体偶置换到全体奇置换的$L_{(12)}$为其逆映射,知其为双射. 从而奇置换与偶置换数目相等.
\end{proof}

\subsection{子群、生成子群}

\par \textbf{3}.
设$G$是群,$g\in G$. 令$C_G(g)=\{x\in G| xg=gx\}$,称为$g$在$G$中的中心化子. 证明:$C_G(g)\le G$,且$Z(G)=\bigcap_{g\in G}C_G(g)$.

\begin{proof}
对$a,b\in C_G(g)$,有$xab=axb=abx$,又由$xa=ax$知$a^{-1}x=xa^{-1}$,从而$ab,a^{-1}\in C_G(g)$, $C_G(g)\le G$. 
\par $Z(G)=\{x\in G| xg=gx, \forall g\in G\}=\bigcap_{g\in G} \{x\in G| xg=gx\}=\bigcap_{g\in G}C_G(g)$.
\end{proof}

\par \textbf{8}.
设$G$是群,$H\le G,K\le G$,证明:$H\cup K \le G\Leftrightarrow H\le K$或$K\le H$.

\begin{proof}
充分性显然,下证必要性. 用反证法,设$H\backslash K \neq \varnothing$且$K\backslash H \neq \varnothing$,取$a\in H\backslash K$,$b\in K\backslash H$. 由$H\cup K$是子群,$a,b\in H\cup K$, 知$ab \in H\cup K$. 若$ab \in H$,推出$b=a^{-1}ab\in H$,矛盾. 若$ab \in K$,同理得$a\in K$,矛盾. 故$H \subseteq K$或$K \subseteq H$,自然有$H\le K$或$K\le H$.
\end{proof}

\subsection{陪集、Lagrange定理}
\par \textbf{7}. 设群$G$非平凡群,且没有非平凡的真子群. 证明: $G$是素数阶循环群.

\begin{proof}
由$G$非平凡群,存在元素$a\neq e$. 考虑$a$的生成子群$\langle a\rangle$,其至少有$a,e$两个元素,由条件只能有$\langle a\rangle=G$,即$G$是循环群. 
\par 若$G$为无限群,则同构于整数加法群,其有非平凡真子群;当$n$是合数阶循环群时,存在阶为$n$素因子的元素,其生成子群为$G$的非平凡真子群. 故$G$只能是素数阶循环群.
\end{proof}

\subsection{正规子群与商群}

\par \textbf{4}.
设$H$是群$G$的子群,$a\in G$,证明:$aHa^{-1}$是$G$的子群且$aHa^{-1}\cong H$.

\begin{proof}
$e=aea^{-1}\in aHa^{-1}$,故$aHa^{-1}$非空;对$h_1,h_2\in H$,$ah_1a^{-1}ah_2a^{-1}=ah_1h_2a^{-1}\in H$, $(ah_1a^{-1})^{-1}=ah^{-1}_1a^{-1}\in H$,故$aHa^{-1}$是子群. 
\par 考虑映射:
\begin{align*}
    \phi:H\rightarrow& \ aHa^{-1}\\
    h\mapsto&  \ aha^{-1}
\end{align*}
则由$aHa^{-1}$的定义知$\phi$是满射;由$\phi(h_1h_2)=ah_1h_2a^{-1}=ah_1a^{-1}ah_2a^{-1}=\phi(h_1)\phi(h_2)$知$\phi$是群同态,又由$aha^{-1}=e \Rightarrow h=a^{-1}a=e$,$\phi$是单射,故$\phi$是群同构.
\end{proof}

\par \textbf{5}. 设$H$是群$G$的子群,证明:若$|H|=k$, 且$G$只有一个$k$阶子群,则$H \unlhd G$.

\begin{proof}
对任意$a\in G$,由2.4.4题知$aHa^{-1}$是$G$的子群且$|aHa^{-1}|=k$. 若$aHa^{-1}\neq H$,与$G$只有一个$k$阶子群矛盾. 故$aHa^{-1}=H$对任意$a\in G$成立,$H$是$G$的正规子群.
\end{proof}

\subsection{同态、同态基本定理}
\subsection{同构定理}
\subsection{群的直积}
\subsection{群在集合上的作用}
\subsection{Sylow定理}